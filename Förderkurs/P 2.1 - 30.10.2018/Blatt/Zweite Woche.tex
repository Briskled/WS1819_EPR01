\documentclass[11pt,a4paper]{article}
\usepackage[latin1]{inputenc}
\usepackage{amsmath}
\usepackage{amsfonts}
\usepackage{amssymb}
\usepackage{graphicx}
\usepackage{listings}
\usepackage{enumitem}
\title{Aufgaben zweite Woche}
\author{Felix Lapp}
\begin{document}
	\maketitle
	\section*{Wiederholung 1: Schreibe einen Algorithmus}
	Schreibe ein Programm, das eine Zahl vom Benutzer entgegennimmt. Nennen wir sie $n$. Anschlie�end soll eine Zahl $Z$ berechnet werden. Mit folgender Formel:
	\begin{align*}
		Z=\frac{n^2-n}{2}
	\end{align*}
	Das Ergebnis $Z$ soll ausgegeben werden.
	
	\section*{Wiederholung 2: Auswertungsreihenfolge}
	Werte folgenden Ausdruck per Hand aus:\\
	
	\begin{center}
		\texttt{(23 - (5 + 2**2 * 2)) * 5 // (5**2) - 4**0}
	\end{center}
	Du kannst das Ergebnis pr�fen, indem du den ganzen Ausdruck in IDLE eingibst. Beispiel f�r eine Auswertung: Es soll \texttt{5**(4-2) + 1} ausgewertet werden. Das kann man so schreiben:\\[5mm]
	\texttt{5**(4-2) + 1}\\
	\texttt{5**(2) + 1}\\
	\texttt{25 + 1}\\
	\texttt{26}
	
	\section*{Aufgabe 1: Datentypen}
	\begin{enumerate}[label=\alph*)]
		\item Finde heraus, welchen Datentypen die \texttt{input}-Funktion zur�ckgibt. Dir werden \texttt{type(...)} und \texttt{help(...)} dabei helfen!\\
		Tipp: \texttt{builtin\_function\_or\_method} ist NICHT die richtige Antwort.
		\item Finde heraus, wie man pr�fen kann, ob es sich bei einer Eingabe um eine ganze Zahl handelt.\\
		\item Was ist der Datentyp \texttt{Boolean}? Was k�nnen die Schl�sselw�rter \texttt{and}, \texttt{or} und \texttt{not}?
		\item Wie l�sst sich pr�fen, ob die Werte von zwei Variablen identisch sind?
	\end{enumerate}

	\section*{Aufgabe 2: Vergleiche}
	Schreibe ein Programm, welches vom Benutzer zwei Eingaben erwartet. Diese Eingaben sollen verglichen werden. Das Ergebnis des Vergleichs soll in einer Variable gespeichert werden. Gebe diese Variable mit \texttt{print} aus.\\
	Kannst du das Programm so modifizieren, dass der Benutzer drei Eingaben zu t�tigen hat und alle drei verglichen werden?
	
	\section*{Live-Programming: if, else, elif}
	Diese Aufgabe kannst du machen wenn du m�chtest. Der Tutor wird sie selbst live f�r alle am Beamer l�sen.\\
	Schreibe ein Programm, welches zwei Eingaben eines Benutzers erwartet. Beide sollen ganze Zahlen sein. Es soll eine Fehlermeldung ausgegeben werden, wenn dies nicht der Fall ist. Anschlie�end soll ausgegeben werden:
	\begin{itemize}
		\item \texttt{Erste Zahl ist gr��er}, wenn die erste Zahl gr��er ist
		\item \texttt{Gleich und Gleich gesellt sich gern}, wenn die Zahlen gleich sind
		\item \texttt{Zweite Zahl ist gr��er}, wenn die zweite Zahl gr��er ist.
	\end{itemize}
	
	
\end{document}